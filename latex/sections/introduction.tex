\documentclass[../template.tex]{subfiles}

\begin{document}

Due to advances in data access, computing power, and statistical techniques, the measurement of latent ideological positions has made significant strides in recent years. This represents a significant and exciting methodological advancement in political science as it enables researchers to control for and explain ideological dimensions in populations where this was previously not possible. A network based approach proposed by Barbera in 2015 is one such technique that has proven robust. In this paper I leverage methodological and computational advances to iterate upon Barbera’s method and propose future research to further improve ideal point estimates. Specifically, I examine the impact of changing the functional form of Barbera’s model to a bilinear model. 
\end{document}