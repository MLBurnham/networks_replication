\documentclass[../template.tex]{subfiles}

\begin{document}
Barbera uses a sample of the twitter accounts belonging to political elites and their followers. A follower network is constructed from this sample such that Yij = 1 if twitter user i follows political elite j and Yij=0 otherwise. The decision to follow a particular elite is modeled as a logistic function of both the elite and users political ideology. Additional controls for the popularity of elite j and political interest of user i are also introduced. The decision of a given use to follow a given elite is thus modeled as such:

\begin{equation} 
P(y _i _j = 1|\alpha _j, \beta _i, \gamma, \theta _i, \phi _j) = logit^-1 (\alpha _j + \beta _i -\gamma || \theta _i - \phi _j ||^2)
\end{equation}

Where \( \mathcal{\alpha}_j \) is the popularity of a given elite, \( \mathcal{B}_i \) is the political interest of a user, \( \mathcal{\theta}_i \) is the ideal point estimate of a given users, \( \mathcal{\phi}_j \) is the ideal point estimate of a given elite, and \( \mathcal{\gamma}\) is a normalizing constant. 

For this research, the most significant assumption in this model is the assumption that a user's decision to follow an elite is a function of the squared euclidean distance of the ideological point estimates of user i and elite j: \( -\mathcal{\gamma}||\mathcal{\theta}_i - \mathcal{\phi}_j||^2 \)

I re-examine this model by instead assuming a bilinear relationship between ideological point estimates and a users decision to follow a political elite: \( -\mathcal{\gamma}(\mathcal{\theta}_i \times \mathcal{\phi}_j)^2 \)

The model I test is thus a slight variation on Barbera's:
\begin{equation} 
P(y _i _j = 1|\alpha _j, \beta _i, \gamma, \theta _i, \phi _j) = logit^-1 (\alpha _j + \beta _i -\gamma (\theta _i \times \phi _j)^2)
\end{equation}


To test this I used Barbera’s original network data from the United States with the euclidian and bilinear model implementations in R’s latentnet package. The data is a set of political actors with twitter accounts circa 2014 that consists of: (1) political representatives in national-level institutions; (2) political parties with twitter accounts; and (3) political media outlets and journalists. Only twitter users with over five thousand followers were used. This results in a total of 318 political elites. From these political elites a sample of 301,537 of their combined followers was taken. The sample was created by subsetting the entire population of their combined followers to those who (1) have sent over one hundred tweets; (2) have sent at least one tweet in the past six months; (3) have more than twenty-five followers; (4) are located inside the borders of the United States; (5) follow at least three political twitter accounts.

\end{document}